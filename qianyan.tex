\chapter*{前言}

车主势,纵横;马主变,制奇。

当对方将位不安、士象不整或者防守子力薄弱时,车马常常能
发起奇袭,古籍中谓之“车马冷招”,这四个字充分体现了车
马配合战术打击的突然性。

车马的突袭之所以可得此美名,原
因大概在于车马这两种子力是天然的进攻拍档。到了残局阶段,盘面
子力减少,空间开阔,为车马威力的发挥创造了条件。车走直
线可纵横如电,寒
气森森,而马走曲线可回环奔跃,威震八面,车马两枚子之间
就可以自然而言形成纵横两路的立体攻势。与之相对的是车需
与天炮和地炮两门重炮才能形成立体攻势。

而且车马之间的配合方式是多姿多彩的,可以组成多种催杀的阵形,如卧槽马、钓鱼
马、高钓马、八角马、高位花心马、低位花心马、篡位马、花
心车等,本书中将车马之间这些基本的催杀阵形一一举例进行
介绍。

在掌握以上基本的攻击阵形的基础上,还需要掌握如何
依据局面在各种催杀阵形之间切换,由于将帅只能在九宫内行
走,而且每次只能走一步,这些都为持车马进攻的一方提供了变换阵
型进攻的诸多可能性,常见的手段包括:马限制将位后车的打将顿挫、
车帅限制黑将活动空间后的绕圈马连续将军同时实现马的大范围转移、拔簧马闪
击变换车位、前马后车阵型借车力变换马位等,而且变换阵型的过程
中常可以顺手牵羊消除黑方的防御力量,
这也是车马冷招令人防不胜防的另一原因。本书中也将针对如
何在各种催杀阵形之间进行切换这一主题举例说明。

准备好了吗?让我们一起来领略车马冷着之美!


